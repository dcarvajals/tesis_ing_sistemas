\setcounter{chapter}{1}
\setcounter{section}{0}
\part{CONTEXTUALIZACIÓN DE LA INVESTIGACIÓN} 

\section{Problema de investigación}

\subsection{Planteamiento del problema}

Con el aumento en la complejidad de los productos de software los desarrolladores de sistemas informáticos han encontrado la manera de mejorar el desarrollo de software, mediante el uso del modelado UML \parencite{Jahan2021}. El modelado de software permite a los desarrolladores comprender todo el diseño de software, obteniendo como resultado una visión general del sistema y una herramienta de comunicación con otros desarrolladores \parencite{gonzalez2022}.

Para empezar con el modelado de un software se necesitan los requisitos planteados por el cliente. Un estudio revelo que el 95\% de los documentos de requisitos de un sistema estaban redactados en algún tipo de lenguaje natural \parencite{Jahan2021}. Todos los requisitos planteados son plasmados en casos de uso, siendo la herramienta de modelado más habitualmente utilizada para representar las especificaciones del software \parencite{hamza2021}. Estos casos de uso permiten analizar más a fondo de forma privada las necesidades del cliente. Sin embargo, al realizar esto suelen surgir confusiones entre el cliente y el desarrollador, debido a que los casos de uso podrían ser modificados, detallando características técnicas del software que el cliente no logrará comprender.

El desacuerdo que existe entre el cliente y el desarrollador podría llevar a resultados poco favorables para ambos. Las descripciones de los casos de uso modificados por el desarrollador podrán ser útiles para construir uno de los diagramas más populares del modelado UML como lo es el diagrama de clases \parencite{Abdelnabi2021}. Pero, para el cliente los casos de uso no tendrán sentido lógico respecto a las condiciones previas dictadas por él.

TDDT4IoTS es una herramienta que usa un lenguaje de símbolos para escribir las descripciones de los casos de uso, permitiendo detallar datos técnicos sobre las clases, interfaces, métodos, etc. que formaran parte del sistema informático a desarrollar. Esta herramienta no cuenta con algún mecanismo que permita notificar si se está utilizando de manera correcta los símbolos respectivos.

Sería importante proveer a los desarrolladores de software o usuarios que utilicen la herramienta TDDT4IoTS, una tecnología que permita detectar los errores cometidos al momento de usar el lenguaje de símbolos para crear los casos de uso extendidos. Además, sería interesante generar una estructura del diagrama de clases pertinente a los casos de uso.

\textbf{Diagnóstico} \\
Analizando el actual interprete que utiliza la herramienta TDDT4IoTS, interpreta solo las descripciones que estén bien escritas. El hecho que no muestra alguna notificación o mensajes indicando algún error que se este cometiendo al momento de redactar los casos de uso, no se sabe como corregir o arreglar lo que este mal escrito. 

\subsection{Formulación del problema}

¿Es posible detectar los errores de escritura en los casos de uso , escritos en el lenguaje de símbolos utilizado en la herramienta TDDT4IoTS para generar un diagrama de clases?

\subsection{Sistematización del problema}

\begin{enumerate}
	\item ¿Qué estructura de dato permitirá a otras librerías que generan diagramas, modificar el diagrama de clases generado por la librería propuesta?
	
	\item ¿Se puede notificar al desarrollador sobre la incorrecta escritura de los casos de uso redactados con el lenguaje de símbolos?
	
	\item ¿Cómo determinar el nivel de efectividad del producto de este proyecto de investigación respecto a la generación del diagrama de clases?
\end{enumerate}

\section{Objetivos}

La problematización de este proyecto ha llevado a plantearse los siguiente objetivos.

\subsection{Objetivo General}

Desarrollar una librería JavaScript que interprete los casos de uso escritos con el lenguaje de símbolos usado en la herramienta TDDT4IoTS, detectando los errores de escritura parar generar un diagrama de clases. 

\subsection{Objetivos Específicos}

\begin{enumerate}
	\item Generar un estructura en formato JSON y XML para almacenar el diagrama de clases, y pueda ser visualizado mediante otras librerías que generen diagramas.
	
	\item Diseñar e implementar una manera de retroalimentar a los usuarios de TDDT4IoTS sobre la incorrecta escritura de los casos de uso redactados con el lenguaje de símbolos. 
	
	\item Evaluar la librería JavaScript propuesta, con la elaboración de diagramas de clases a partir de casos de uso extendidos correspondientes a requisitos de sistemas de información.
\end{enumerate}

\section{Justificación}

Los requisitos planteados por el cliente no siempre estarán claros desde el principio \parencite{iqbal2020}. Al momento de comenzar la fase de desarrollo de un sistema informático, pueden surgir problemas que deberán ser resueltos de diferentes formas a las que fueron planteadas al inicio. Existen herramientas que permiten crear los diagramas de casos de uso transmitiendo de forma gráfica los requisitos que se deben ejecutar, pero no detallan toda la información requerida por un solo caso de uso \parencite{Abdelnabi2021}.

Toda la información de un caso de uso detalla las condiciones previas, precondiciones y secuencia de eventos que deberá seguir un sistema informático \parencite{iqbal2020}. Es decir, toda esa información permitirá generar otros tipos de diagramas usando el modelado UML. Sin embargo, las herramientas existentes hasta el momento permiten generar varios tipos de diagramas UML, pero es necesario para poder crearlos o modificarlos, construirlos por separado y de manera manual.

Una solución para mejorar el rendimiento de los desarrolladores es utilizar un lenguaje para la escritura de los casos de uso que permita escribir toda la información necesaria de un caso de uso y al mismo tiempo permita detallar información técnica del sistema informático. Se podrá obtener todo lo necesario para crear un diagrama de casos de uso y generar de forma automática la estructura del diagrama de clases pertinente a los datos técnicos detallados con la ayuda del lenguaje de símbolos usando por la herramienta TDDT4IoTS. 

Para los usuarios de la herramienta TDDT4IoTS será muy beneficioso contar con una librería que le permita interpretar para detectar si la escritura de la información es correcta. Además, el producto final de este trabajo permitirá modificar los casos de uso y al mismo tiempo poder hacerlo con los datos técnicos referentes al diagrama de clases, tratando de reducir el tiempo en el refinamiento de modelos del software. Finalmente se puede mencionar que no se ha encontrado aplicaciones o algún tipo de software que permita escribir los casos de uso, haciendo posible la generación de otros tipos de diagramas. 