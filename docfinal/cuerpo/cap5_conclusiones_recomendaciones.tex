\setcounter{chapter}{5}
\setcounter{section}{0}
\part{CONCLUSIONES Y RECOMENDACIONES}

\section{Conclusiones}

\begin{itemize}
	\item Con la utilización de un lenguaje de símbolos que permita estar combinado con las descripciones de los casos de uso escritos en texto natural, ademas que permita detallar datos técnicos que intervendrán en la generación del sistema informático, logra disminuir un poco el tiempo en la creación de este diagrama de clases. Previamente el desarrollador podrá utilizar la librería en su propio proyecto o utilizar la aplicación web de demostración.
	\item Para los usuarios que utilizan la herramienta TDDT4IoTS al momento de interpretar las descripciones de los casos de uso con la librería desarrollada, les permite identificar de forma mas rápida donde están cometiendo errores al momento de utilizar los símbolos respectivos.
	\item La aplicación de demostración sobre como funciona la librería servirá para los desarrolladores que necesiten generar un diagrama de clases de una forma rápida dependiendo del sistema informático al que se estén enfrentando. La librería puede ser descargada para que cada desarrolladores personalice la forma en como y con que genere el diagrama de clases, obteniendo una aplicativo que funcione con todos los proyectos que desarrolle a futuro.   
	\item Se logro obtener un solo archivo de tipo .js que puede ser exportado a cualquier tipo de proyecto web. Esto beneficia a muchas aplicaciones que necesiten de la interpretación de los casos de uso que usen los símbolos de una forma rápida y sin tener que instalar programas externos para su correcto funcionamiento.
\end{itemize}

\newpage
\section{Recomendaciones}
	
\begin{itemize}
	\item La librería desarrollada en el lenguaje javascript puede ser fácilmente escalable. Se puede desarrollar un proyecto npm y subirlo al repositorio oficial para que pueda ser utilizado mediante proyectos que utilicen como servidor de aplicaciones Node.js. 
	\item Durante la elaboración de la retroalimentación sobre como utilizar los símbolos para escribir las descripciones de los casos de uso, se pueden realizar trabajos posteriores sobre como identificar la posición exacta del error que se esta cometiendo. Se pueden utilizar una serie de colores que permita identificar los caracteres donde se encuentren las anomalías.
	\item Para lograr que la librería pueda ser mejorada, se puede pensar en publicar el código fuente en una plataforma web como lo es GitHub y permitir que otros desarrolladores  mejoren la librería para llegar a un mayor alcance en la comunidad de programadores a nivel mundial.
	\item Para lograr un mayor alance sobre las tecnologías de desarrollo que existen actualmente, y con las que en un futuro serán desarrolladas. Se podrá crear un proyecto que este desplegado en un servidor que este en la nube para crer un servicio web que realice el mismo proceso de la librería, con la ventaja que podra ser utilizado en casi todos los tipos de proyectos que existen. Se podrá obtener los resultados del proyecto de forma clara y remota.  
\end{itemize}