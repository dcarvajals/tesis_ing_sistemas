\setcounter{chapter}{2}
\setcounter{section}{0}
\part{FUNDAMENTACIÓN TEÓRICA DE LA INVESTIGACIÓN}

En este capítulo, se expondrán los trabajos relacionados con el presentado en este documento que se han identificado en la literatura, y que demuestran que el trabajo propuesto en este documento, aporta algo. Además se contextualiza el trabajo y define los términos novedosos y de poco dominio para los investigadores y para la comunidad.

\section{Marco conceptual.}

	En esta sección, se detalla los modelos teóricos, conceptos, argumentos o definiciones que se han desarrollado o investigado en relación con el tema en particular.

\subsection{JSON.}

Javascript Object Notation (JSON) es un formato ligero de intercambio de datos. Consisten en asociación de nombres y valores. A pesar de ser independiente del lenguaje de programación, es admitido en una gran cantidad de lenguajes de programación. Se basa en un subconjunto del Estándar de lenguaje de programación JavaScript \cite{JSON}.

\subsection{XML.}

Es un lenguaje de marcado similar a HTML. Significa Extensible Markup Language y pertenece a la especificación W3C como lenguaje de marcado de propósito general. Esto significa que, a diferencia de otros lenguajes de marcado, XML no está predefinido, por lo que debe definir su propio marcado. El objetivo principal del lenguaje es compartir datos entre diferentes sistemas, como Internet \cite{XML-based}.

\subsection{Compiladores.}
Están diseñados para traducir un fragmento de código escrito en un de lenguaje de programación a lenguaje de máquina que es, el que puede entender la computadora. El compilador analiza el código fuente en busca de errores antes de la traducción. Si se detecta un error, el compilador
notificar al autor del código para que pueda arreglarlo.. Además, los compiladores modernos o entornos de desarrollo (IDE), puede sugerir soluciones para algunos tipos de errores usando métodos de corrección de errores \cite{CoEdit}.

\subsection{Lenguajes de modelado.}
Representan una serie de requisitos basados en la construcción de elementos visuales para definir estructuras y comportamientos que tendrán los sistemas computarizados. UML (Lenguaje Unificado de Modelado) a través del mecanismo de perfilado, se han basado históricamente en notaciones gráficas. UML mediante el mecanismo de perfiles, maximiza la comprensión humana y facilita la comunicación entre las partes interesadas como son el cliente y desarrollador \cite{Blended}. 

También existen lenguajes de modelado personalizados para distintas áreas, como por ejemplo en \cite{Multi-level} proponen un lenguaje de modelado conceptual multinivel al denominan ML2 (Lenguaje de Modelado Multinivel). El lenguaje está orientado al modelado conceptual multinivel (de dominio) y pretende cubrir un amplio conjunto de dominios multiniveles. En el diseño de ML2 sigue un enfoque basado en principios, definiendo su sintaxis abstracta para reflejar una teoría formal para el modelado multinivel que se fue desarrollado previamente.

\section{Marco referencial.}

\subsection{Lenguaje de restricciones de objetos para la generación de código a partir de modelos de actividad}
UML(Unified Modeling Language) es un lenguaje que utiliza el diagrama de actividad para modelar el flujo de trabajo y el flujo de objetos en un sistema \cite{Improving}. UML no es un lenguaje totalmente formal, su semántica no está totalmente formalizada ocasionando un escenario donde la presentación precisa del modelo es difícil. Por lo tanto, siempre que se utilicen diagramas de actividades, o cualquier diagrama UML, para la generación de código, se recomienda complementarlo con lenguajes de especificación como el lenguaje de restricciones de objetos OCL (Object Constraint Language) \cite{Object}.   

\subsection{Desarrollo de módulos de software generativo para el diseño orientado al dominio con un lenguaje específico de dominio basado en anotaciones}
En la terminología del diseño orientado a objetos \cite{Feature}, un módulo de dominio es un paquete. Los actuales marcos de software DDD (Object-oriented domain-driven design) han utilizando una forma simple de DSL(Domain-specific language) interno para construir el modelo de dominio y utilizar este modelo como entrada para generar un prototipo de software. El DSL interno que utilizan se conoce más formalmente como lenguaje específico del dominio basado en anotaciones (aDSL) \cite{Generative}. 

\subsection{Modelado para la localización de características en modelos de software: tanto la generación de código y los modelos interpretados.}
Evaluando LDA (Latent Dirichlet Allocation), cada caso de estudio utiliza un tipo diferente de modelos de software: modelos de software para la generación de código, y modelos de software para interpretación. El primer caso de estudio pertenece a un líder mundial en fabricación de trenes, construcciones y Auxiliar de Ferrocarriles. Una empresa que formaliza los productos fabricados en modelos de software utilizando un lenguaje específico de dominio DSL. Los modelos de software se utilizan para generar el firmware que controla sus trenes \cite{Topic}. 

El segundo estudio de caso pertenece a un videojuego comercial, Kromaia, que utiliza modelos de software para razonar sobre el sistema, realizar validaciones y definir el contenido del juego, como los jefes, los mundos y los objetivos. En Kromaia los modelos de software se utilizan para la interpretación. Así, el contenido definido en los modelos se lee e interpreta cuando se lanza el juego (sin alterar el código fuente del videojuego). El videojuego se ha lanzado en todo el mundo en dos plataformas diferentes (PlayStation 4 y STEAM) y en 8 idiomas diferentes. \cite{Topic}.

\section{Marco referencial.}