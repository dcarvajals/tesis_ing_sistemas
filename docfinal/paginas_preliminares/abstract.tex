\begin{center}
	{\titulodc \textbf{ABSTRACT AND KEYWORDS}}
\end{center}

Software modeling is used to create an abstract idea of everything necessary to develop a computer system. There are modeling languages that are used to perform this task, but a certain number of developers comment that this phase in software development is very tedious because of the documentation that must be done before generating the source code. One of the most used diagrams to represent software is the class diagram. This diagram allows to detail all the objects that will be necessary for the final product to meet the customer's expectations. To take the software requirements there is another type of tool that is very useful to perform this task, the tool is called use cases. These are used to write down everything that the client mentions about the operation of the software to be developed. The developer should pay attention to everything the customer needs for the system. The use cases are the most convenient to write down the needs or requirements of the system. From everything mentioned by the client that will use the system, the analyst begins to recognize the objects that must be generated for the subsequent development of the software. That is why the objective of this work is to create a JavaSript library that allows to interpret the descriptions of the use cases written with a symbol language that uses the TDDT4IoTS tool to mark words that represent elements of object-oriented modeling and generate the structure of a class diagram in JSON and XML format, so that it can then be manipulated by other specific web libraries to draw other types of diagrams, but with the structure generated by this library can get the class diagram in an automatic way.  

\textbf{Keywords:} Case tools, class diagrams, use case specification, use case specification language, software engineering.