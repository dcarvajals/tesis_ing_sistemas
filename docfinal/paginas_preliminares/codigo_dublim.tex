\begin{center}
	{\LARGE \textbf{CÓDIGO DUBLIN}}
\end{center}

\begin{table}[h!]
	\begin{tabular}{| p{2.2cm} | p{12cm} |}
		\hline
		Titulo: & Librería JavaScript que permita interpretar la escritura de los casos de uso extendidos. \\ \hline
		Autores: & Carvajal Suárez, Dúval Ricardo \\ \hline
	\end{tabular}
	\begin{tabular} {| p{2.2cm} | p{2.1cm} | p{1.5cm} | p{2.6cm} | p{2.1cm} | p{1.96cm} |}  
		Palabras claves: & Herramienta case & Diagrama de clases & Especificación de casos de uso  & SymLen de casos de uso & Ingeniería de software \\ \hline
	\end{tabular}
	\begin{tabular}{| p{2.2cm} | p{12cm} |}
		Fecha de publicación: & Diciembre, 2022 \\ \hline
		Director del proyecto: & Guerrero Ulloa, Gleiston Cicerón  \\ \hline
		Editorial: & Quevedo: UTEQ 2022  \\ \hline
		Resumen: &  {\small Resumen - El modelamiento de software ayuda a visualizar los elementos necesarios para desarrollar un sistema informático. Existen lenguajes de modelado que sirven para realizar esta tarea, sin embargo, algunos desarrolladores comentan que esta fase en el desarrollo de software es muy tediosa, ya que, se debe documentar y validar el software antes de generar el código fuente. Uno de los diagramas más utilizados para visualizar el sistema informático es el diagrama de clases. Este diagrama permite detallar todas las clases de objetos que serán necesarios para que el producto final cumpla con las expectativas del cliente. Para tomar los requisitos del sistema y del software existen otras herramientas, entre ellas los casos de uso y las historias de usuario. Los casos de uso sirven para anotar todo lo que el cliente mencione sobre el funcionamiento del software a realizar mediante una secuencia de acciones. El desarrollador deberá prestar atención a todo lo que el cliente necesite realizar utilizando el sistema. Los casos de uso son los más convenientes para redactar las necesidades del cliente o requisitos del sistema.} \\
		& \textbf{{\small  Abstract - Software modeling helps to visualize the elements necessary to develop a computer system. There are modeling languages that are used to perform this task, however, some developers comment that this phase in software development is very tedious, since the software must be documented and validated before generating the source code. One of the most commonly used diagrams to visualize the computer system is the class diagram. This diagram allows to detail all the classes of objects that will be necessary for the final product to meet the customer's expectations. Other tools exist to take the system and software requirements, including use cases and user stories. Use cases are used to write down everything that the customer mentions about the operation of the software to be performed through a sequence of actions. The developer should pay attention to everything the customer needs to perform using the system. The use cases are the most convenient to write down the customer's needs or system requirements.}} \\
		\hline
		Descripción: & 116 hojas, dimensiones: 21cm x 29.7cm  \\ \hline
		URI: &   \\ \hline
	\end{tabular}
\end{table}

% Herramientas case - diagrama de clases - especificación de casos de uso - SymLen de casos de uso & ingeniería de software