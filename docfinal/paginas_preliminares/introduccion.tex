\begin{center}
	\textbf{ {\titulodc Introducción}}
\end{center}

Los sistemas informáticos se están volviendo cada vez más indispensables en todos los aspectos de la vida cotidiana. Considerando que la tecnología cambia aceleradamente, y que los usuarios requieren soluciones rápidas, la Ingeniería de Software busca disminuir el tiempo en implementar los sistemas requeridos \parencite{Panthi2022}. Para la construcción de una aplicación informática que cumpla con los requisitos del cliente y que sea eficiente se necesitan aplicar varios aspectos técnicos, que la Ingeniería de Software detalla \parencite{Chen2022}. Por ejemplo: estilo arquitectural, tecnologías a utilizar, infraestructura, entre otros \parencite{Kulesza2020}.

El campo de la Ingeniería de Software es análogo al campo de la construcción de obras civiles, entre otros. Los documentos de diseño de un Software guardan semejanza con los documentos de diseño de una obra civil (por ejemplo, un edificio). Para la construcción de un edificio se deben analizar las principales características físicas del terreno sobre el cual se va a construir, mientras que, para la construcción de un software se debe analizar la/s plataforma/s sobre la/s cual/es se ejecutará el software. En el caso de un edificio, se deben detallar aspectos de la construcción como los ambientes y sus dimensiones precisas, columnas, puertas, ventanas, etc. Así mismo, en los sistemas informáticos se necesita documentar con precisión los módulos, sus funcionalidades y sus alcances, y otros elementos que conformarán el sistema a desarrollar. Además del comportamiento que deben tener los artefactos de un software \parencite{Hamdi2022}. Lo que se puede mencionar como diferencias entre estos dos campos es que los elementos de software pueden ser construidos para ser reutilizados en otros sistemas \parencite{Guerra2021}.

\textit{Unified Modeling Language} (UML) con su traducción al español Lenguaje Unificado de Modelado. Es uno de los lenguajes más conocidos por los desarrolladores de software para la representación de los requisitos o necesidades del usuario \parencite{OMGUML24}. Sin embargo, estos diagramas no son fáciles de comprender para todos los usuarios, ya que, los elementos que intervienen y que darán solución al problema planteado son representados mediante símbolos y algo de texto. UML se utiliza específicamente en la industria del software para especificar, visualizar, construir y documentar los artefactos de un sistema de software \parencite{Bergstrom2022}. UML se encuentra definido oficialmente por el \textit{Object Management Group} (OMG) con su traducción al español Grupo de Administración de Objetos \parencite{Omg2009}. Algunos investigadores afirman que, a partir de modelos dinámicos (diagramas de casos de uso) y estáticos (diagramas de clases) se pueden generar otros modelos UML de manera más eficiente \parencite{Jahan2021}, y así obtener software de calidad.

Los estándares ISO/IEC 9126-1, especifican la calidad del software tanto interna, externa y en uso del producto. De ahí la importancia de considerar seguir estándares que permitan asegurar un lenguaje común para el entendimiento de todos los involucrados en el proyecto de software \parencite{Losavio2009}.

Los diagramas de casos de uso son una representación del sistema mediante los actores (nota al pie) y sus requisitos (casos de uso). Son una herramienta para representar los requisitos iniciales del sistema, y pueden ser fácilmente entendidos por todos los involucrados \parencite{Zapata2008}. Sin embargo, en un diagrama de casos de uso no se especifican los detalles, sino que son los requisitos a \textit{grano grueso} del sistema. 

Para mayor detalle de los requisitos se utilizan los casos de uso extendidos, que detallan los requisitos a \textit{grano fino} del sistema, y se plasman en documentos específicos. Estos requisitos de grano fino pueden ser comprimidos en los diagramas de clases \parencite{Abdelnabi2021}. Los casos de uso tienen algunos elementos de información, entre ellos: 

\begin{itemize}
	\item \textbf{Descripción:} describen de forma textual varias formas que los actores pueden trabajar con el software.
	\item\textbf{ Precondiciones:} son las condiciones que debe cumplir el sistema para qué sé escriban en el caso de uso.
	\item \textbf{Poscondiciones:} es el estado en que el sistema se encuentra después de haber realizado el caso de uso. Se deben considerar al momento de estar utilizando el software \parencite{iqbal2020}.
	\item \textbf{Secuencia normal de eventos:} sirve para describir las acciones de los actores y las respuestas del sistema en forma cronológica.
	\item \textbf{Secuencia alternativa de eventos:} describen las respuestas del sistema y las acciones de los actores cuando se dan ciertas condiciones.
\end{itemize}

\par{
Los diagramas de clases en UML representan la estructura estática de los objetos y sus posibles conexiones dentro del software. Se lo utiliza para ilustrar el punto de vista estático, exponiendo un conjunto de clases, interfaces y relaciones \parencite{abu2020}. Se lo desarrolla durante la fase de elaboración y se perfecciona posteriormente en la fase de construcción \parencite{Omg2009}, mostrando un modelo del dominio del sistema. Además, es uno de los diagramas más útiles en UML, ya que trazan claramente la estructura de un sistema concreto a modelar \parencite{abu2020}.}

Los diagramas de clases también son implementados en muchas herramientas de modelado y utilizados en la generación automática de software, entre las que se puede citar a Rational Rose, MagicDraw, ArgoUML, por mencionar algunas. Sin embargo, ninguna considera la documentación de los casos de uso, por lo tanto, tampoco emplean las especificaciones a grano fino para obtener diagramas UML que ayuden a generar el código como lo hace el diagrama de clases.

Existe una herramienta TDDT4IoTS \parencite{tddt4iots} que está en fase de prueba. En esta herramienta es posible que el analista escriba los requisitos detallados de los casos de uso, y, mediante el uso de un lenguaje de símbolos (SymLen) marcar las palabras que representan nombres de clases, atributos, métodos, interfaces y otros elementos de los diagramas de clases. A su vez, presenta los casos de uso en lenguaje natural con las palabras que el usuario utilizó para expresar los requisitos detallados que el sistema debe cumplir. De estas especificaciones de grano fino es posible generar los diagramas de clases, y el código de software correspondiente. Al ser una herramienta en su fase inicial, el intérprete implementado en TDDT4IoTS no presenta una retroalimentación eficiente, simplemente presenta los artefactos que son posible generar según la escritura, sin especificar los posibles errores en el uso del lenguaje. Además, el actual intérprete no se puede utilizar fácilmente en otra herramienta.

En el presente trabajo se presenta una librería escrita en JavaScript denominada \textit{Armadillo} (Armadillo.js), que permite interpretar las especificaciones de los casos de uso escritas en SymLen. Armadillo le permite al analista conocer los posibles errores en el uso de SymLen, con el objetivo de obtener de forma automática un diagrama de clases pertinente a la información proporcionada por el usuario. Esto permitirá potenciar el uso de la herramienta TDDT4IoTS. Además, Armadillo le permite eliminar los inconvenientes del intérprete que actualmente utiliza TDDT4IoTS que han sido detectados.

\newpage

El resto de este documento está organizado por capítulos. En el capítulo I se contextualiza la investigación, donde se especifica la problemática a resolver, los objetivos que se proponen lograr y la hipótesis a demostrar, además de una argumentación que justifica el porqué de esta investigación. En el capítulo II se presenta la fundamentación teórica, en la que se detallan los principales trabajos relacionados con este documento. Además, el alcance y las limitaciones, las herramientas y tecnologías a utilizar, y los conceptos y definiciones importantes. En el capítulo III se detalla la metodología de investigación aplicada al proyecto para llevar a cabo un orden sobre las acciones que se realizaron a lo largo del desarrollo del proyecto. En el capítulo IV se especifican los resultados que se obtuvieron al momento de implementar las funcionalidades del proyecto presentado. En el capítulo V se redactan las conclusiones y recomendaciones que se analizaron sobre todo lo trabajado y finalmente en el capítulo VI se visualiza el listado de la bibliografía usada para el proyecto. 



