\begin{center}
	\textbf{ {\titulodc Introducción}}
\end{center}

Los sistemas informáticos se están volviendo cada vez más indispensable en todos los aspectos de la vida cotidiana. Por ende, los desarrolladores de software buscan disminuir el tiempo que toma desarrollar diferentes tipos de aplicaciones \parencite{Panthi2022}. Para la construcción de una aplicación informática se necesitan aplicar varios conceptos técnicos para su correcto funcionamiento \parencite{Chen2022}. Al igual que el plano de un edificio se detallando las principales características físicas del terreno, columnas, puertas, ventanas, etc. Se necesita documentar con precisión el comportamiento que deben tener todos los artefactos de un software y de ser necesario que sean reutilizables \parencite{Hamdi2022}.

\textit{Unified Modeling Languag}e (UML) con su traducción al español Lenguaje Unificado de Modelado, se utiliza específicamente en la industria del software para especificar, visualizar, construir y documentos los artefactos de un sistema de software \parencite{Bergstrom2022}. UML se encuentra definido oficialmente por el \textit{Object Management Group} (OMG) con su traducción al español Grupo de Administración de Objetos \parencite{Omg2009}. Algunos investigadores han propuesto que la mejor forma de generar modelos UML es a partir de modelos estáticos como los diagramas de casos de uso y diagramas de clases \parencite{Jahan2021}.

Los diagramas de casos de uso representan al sistema mediante los usuarios (actores) y sus requisitos (casos de uso). La información de cada caso de uso detalla las condiciones previas, precondiciones y secuencia de eventos. Además, incluye la secuencia alternativa de eventos en caso de excepciones o condiciones específicas y las postcondiciones que se deben tomar en cuenta al momento de estar utilizando el software \parencite{iqbal2020}. Existe la forma de comprimir globalmente los requisitos de un sistema, una de las herramientas más populares para poder realizarlo son los diagramas de clases \parencite{Abdelnabi2021}. 

En UML los diagramas de clases representan la estructura estática de los objetos y sus posibles conexiones dentro del software. Se lo utiliza para ilustrar el punto de vista estático, exponiendo un conjunto de clases, interfaces y relaciones \parencite{abu2020}. Se lo desarrolla durante la fase de elaboración y se perfecciona posteriormente en la fase de construcción \parencite{Omg2009}, representando el modelo del dominio del sistema. Además, es uno de los diagramas más útiles en UML, ya que trazan claramente la estructura de un sistema concreto al modelar sus clases, atributos, operaciones y relaciones entre clases \parencite{abu2020}.

En la herramienta TDDT4IoTS \parencite{tddt4iots} es posible escribir las descripciones de los casos de uso usando un lenguaje de símbolos, pero es necesario verificar si los casos de uso estan escritos de manera correcta. Este trabajo tiene como idea, desarrollar una librería escrita en el lenguaje de programación JavaScript denominada \textbf{Armadillo.js}. Armadillo permitirá interpretar las descripciones de los casos de uso extendidos, escritos en un lenguaje de símbolos usado por la herramienta TDDT4IoTS. Finalmente se obtiene como resultado la estructura de un diagrama de clases pertinente a la información obtenida por los casos de uso de forma automática.

