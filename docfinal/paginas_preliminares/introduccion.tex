\begin{center}
	\textbf{ {\titulodc Introducción}}
\end{center}

Los sistemas informáticos se están volviendo cada vez más indispensables en todos los aspectos de la vida cotidiana. Considerando que la tecnología cambia aceleradamente, y que los usuarios requieren soluciones rápidas, son las principales razones por las que la Ingeniería de Software busca disminuir el tiempo en implementar los sistemas requeridos \parencite{Panthi2022}. Para la construcción de una aplicación informática que cumpla con los requisitos del cliente y que sea eficiente se necesitan aplicar varios aspectos técnicos \parencite{Chen2022}, que la Ingeniería de Software detalla. Por ejemplo: estilo arquitectural, tecnologías a utilizar, infraestructura, escalabilidad, entre otros \parencite{Kulesza2020}.

El campo de la Ingeniería de Software es análogo al campo de las construcción de obras civiles. Los documentos de diseño de un Software guardan semejanza con los documentos de diseño de un edificio (por ejemlo). Para la construcción de un edificio se deben analizar las principales características físicas del terreno sobre el cual se va a construir, mientras que, para la construcción del software se debe analizar la/s plataformas sobre las cuales se ejecutará el software. En el caso de un edificio, se deben detallar aspectos de la construcción ambientes y sus dimensiones precisas, columnas, puertas, ventanas, etc., así mismo se necesita documentar con precisión los módulos y las funcionalidades y alcance de cada uno de ellos, los elementos que conformarán el producto de software a desarrollar, además del comportamiento que deben tener todos los artefactos de un software \parencite{Hamdi2022}. Lo que se puede mencionar como diferencias que estos elementos de software pueden ser construidos para ser reutilizados en otros sistemas \parencite{Guerra2021}.

\textit{Unified Modeling Languag}e (UML) con su traducción al español Lenguaje Unificado de Modelado, Uno de los lenguajes más conocidos por los desarrolladores de software para la representación de los requisitos o necesidades del cliente/usuario final (en adelante solamente usuario) son los diagramas UML \parencite{OMGUML24}. Sin embargo, estos diagramas no son fáciles de comprender para todos los usuarios, ya que, los elementos que intervienen y que darán solución al problema planteado son representados mediante símbolos y algo de texto. UML se utiliza específicamente en la industria del software para especificar, visualizar, construir y documentar los artefactos de un sistema de software \parencite{Bergstrom2022}. UML se encuentra definido oficialmente por el \textit{Object Management Group} (OMG) con su traducción al español Grupo de Administración de Objetos \parencite{Omg2009}. Algunos investigadores afirman que, a partir de modelos dinámicos (diagramas de casos de uso) y estáticos (diagramas de clases) se pueden generar otros modelos UML de manera más eficiente \parencite{Jahan2021}, y así obtener software de calidad.

Los estándares ISO/IEC 9126-1, especifican la calidad del software tanto interna, externa y en uso del producto. De ahí la importancia de considerar seguir estándares que permitan asegurar un lenguaje común para el entendimiento de todos los involucrados en el proyecto de software \parencite{Losavio2009}.

Los diagramas de casos de uso son una representación del sistema mediante los usuarios (actores) y sus requisitos (casos de uso). Son una buena herramienta para representar los requisitos iniciales del sistema, y pueden ser fácilmente entendidos por todos los involucrados \parencite{Zapata2008}. Sin embargo en un diagrama de caso de uso no se especifican los detalles sino que son los requisitos a \textit{grano grueso} del sistema. La información de cada caso de uso se detalla mediante condiciones previas (precondiciones), poscondiciones y secuencia de eventos normales. Además, incluye la secuencia alternativa de eventos en caso de excepciones o condiciones específicas y las postcondiciones que se deben tomar en cuenta al momento de estar utilizando el software \parencite{iqbal2020}. Estas especificaciones de los casos de uso se denominan los requisitos a \textit{grano fino}, y se plasman en documentos denominados \textit{casos de uso extendidos}. Estos requisitos de grano fino pueden ser comprimidos en los diagramas de clases \parencite{Abdelnabi2021}. 

Los diagramas de clases en UML representan la estructura estática de los objetos y sus posibles conexiones dentro del software. Se lo utiliza para ilustrar el punto de vista estático, exponiendo un conjunto de clases, interfaces y relaciones \parencite{abu2020}. Se lo desarrolla durante la fase de elaboración y se perfecciona posteriormente en la fase de construcción \parencite{Omg2009}, representando un modelo del dominio del sistema. Además, es uno de los diagramas más útiles en UML, ya que trazan claramente la estructura de un sistema concreto al modelar sus clases con sus atributos y operaciones, y las relaciones entre clases \parencite{abu2020}.

Los diagramas de clases son utilizados por muchas herramientas (Rational Rose\footnote{https://www.ibm.com/docs/es/rsas/7.5.0?topic=migration-rational-rose-model}, MagicDraw\footnote{https://www.magicdraw.com/}, ArgoUML\footnote{https://argouml-tigris-org.github.io/}, por mencionar algunas) para la generación del código de software a partir de los diagramas de clases, sin embargo ninguna considera la documentación de los casos de uso extendidos, por lo tanto, tampoco utilizan las especificaciones a grano fino para obtener diagramas UML que ayudan a generar el código como lo es el diagrama de clases.

Existe una herramienta TDDT4IoTS \parencite{tddt4iots} que está en fase de prueba. En esta herramienta es posible que el analista escriba los requisitos detallados de los casos de uso, y, mediante el uso de un lenguaje de símbolos (SymLen) marcar las palabras que representan nombres de clases, atributos, métodos, interfaces y otros elementos de los diagramas de clases. A su vez, presenta los casos de uso extendidos en lenguaje natural exactamente como el usuario expresó los requisitos detallados que el sistema debe cumplir. De estas especificaciones de grano fino es posible generar los diagramas de clases, y el código de software correspondiente. Al ser una herramienta en su fase inicial, el interprete implementado en TDDT4IoTS no presenta una retroalimentación eficiente, simplemente presenta los artefactos resultado, sin especificar los posibles errores en el uso del lenguaje. Además el actual intérprete no se puede utilizar fácilmente en otra herramienta.

En el presente trabajo se propone una librería escrita en JavaScript denominada \textit{Armadillo} (Armadillo.js). Armadillo permite interpretar las descripciones de los casos de uso extendidos escritas en SymLen. Armadillo le permite al analista conocer los posibles errores en el uso de SymLen, con el objetivo de obtener de forma automática un diagrama de clases pertinente a la información proporcionada por el usuario. Esto permitirá potenciar el uso de la herramienta TDDT4IoTS. Además, Armadillo le permitirá eliminar todos los inconvenientes del intérprete que actualmente utiliza TDDT4IoTS que han sido detectados.

El resto de este documento está organizado por capítulos. En el capítulo I se contextualiza la investigación, donde se especifica la problemática a resolver, los objetivos que se lograron y la hipótesis que se demostró, además de una argumentación que justifica que se haya realizado esta investigación. En el capítulo II se presenta la fundamentación teórica, en la que se presentan los principales trabajos relacionados con el trabajo propuesto en este documento, conceptos y definiciones importantes, y las limitaciones, herramientas y tecnologías se expresan en el marco contextual de la investigación.

