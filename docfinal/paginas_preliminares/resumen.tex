\begin{center}
	{\titulodc \textbf{RESUMEN Y PALABRAS CLAVES}}
\end{center}

El modelamiento de software sirve para crear una idea abstracta sobre todo lo que se quiere lograr con un sistema informático. Existen lenguajes de modelado que sirven para realizar esta tarea, pero para la mayor parte de desarrolladores esta fase en le desarrollo de software es muy tediosa por la documentación que se debe realizar antes de generar el código fuente. Uno de los diagramas mas utilizados para representar totalmente un software es el denominado diagrama de clases, el cual permite detallar todos los objetos que serán necesarios para que el producto final cumpla con las expectativas del cliente. Para tomar los requisitos del cliente existe otro tipo de diagrama que es muy útil para realizar esta tarea que son los diagramas de casos de uso, el cliente puede mencionar todo lo que el software realice, ya se de forma verbal o escrita. El desarrolladores deberá prestar atención a todo lo que el cliente necesita para su sistema, los casos de uso son los mas convenientes para redactar las necesidades o requisitos que el cliente ha mencionado. A partir de los requisitos mencionados por el usuario final que utilizare el sistema se empieza a reconocer todos los objetos que se deben generar para el software. Por eso el objetivo de este trabajo es crear una librería javascript que permita interpretar las descripciones de los casos de uso escritas con un lenguaje de símbolos que usa la herramienta TDDT4IoTS para detallar datos técnicos sobre los objetos del software, y generar la estructura de un diagrama de clases en formato JSON y XML para que luego puedo ser manipulado mediante otras librerías web que permitan dibujar otros tipos de diagramas, pero con la estructura que genera esta librería pueda conseguir le diagrama de clases de una forma mas rápida y correcta. 

\textbf{Palabras claves:} generación de diagramas, detector de errores, modelado uml, desarrollo ágil, diagrama de clases
