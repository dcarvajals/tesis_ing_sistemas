\begin{center}
	{\titulodc \textbf{RESUMEN Y PALABRAS CLAVES}}
\end{center}

El modelamiento de software ayuda a visualizar los elementos necesarios para desarrollar un sistema informático. Existen lenguajes de modelado que sirven para realizar esta tarea, sin embargo, algunos desarrolladores comentan que esta fase en el desarrollo de software es muy tediosa, ya que, se debe documentar y validar el software antes de generar el código fuente. Uno de los diagramas más utilizados para visualizar el sistema informático es el diagrama de clases. Este diagrama permite detallar todas las clases de objetos que serán necesarios para que el producto final cumpla con las expectativas del cliente. Para tomar los requisitos del sistema y del software existen otras herramientas, entre ellas los casos de uso y las historias de usuario. Los casos de uso sirven para anotar todo lo que el cliente mencione sobre el funcionamiento del software a realizar mediante una secuencia de acciones. El desarrollador deberá prestar atención a todo lo que el cliente necesite realizar utilizando el sistema. Los casos de uso son los más convenientes para redactar las necesidades del cliente o requisitos del sistema. A partir de todo lo mencionado por el usuario que utilizará el sistema, el desarrollador empieza a reconocer los objetos que se deben generar para la posterior implementación (construcción) del software. Por eso, el objetivo de este trabajo es crear una librería JavaScript denominada Armadillo que permita interpretar  los casos de uso escritos con un lenguaje de símbolos que utiliza la herramienta TDDT4IoTS (\url{https://aplicaciones.uteq.edu.ec/tddt4iots/}) para marcar palabras que representan elementos del modelamiento orientado a objetos y generar la estructura de un diagrama de clases en formato JSON y XML, para que luego pueda ser manipulado utilizando otras librerías web para dibujar diagramas de clases y cualquier otro tipos de diagrama a partir de este.

\textbf{Palabras claves:} Herramienta case, diagrama de clases, especificación de casos de uso, lenguaje para especificación casos de uso, ingeniería de software.
