\begin{center}
	{\titulodc \textbf{RESUMEN Y PALABRAS CLAVES}}
\end{center}

El modelamiento de software sirve para crear una idea abstracta sobre todo lo necesario para desarrollar un sistema informático. Existen lenguajes de modelado que sirven para realizar esta tarea, pero una cierta cantidad de desarrolladores comentan que esta fase en el desarrollo de software es muy tediosa por la documentación que se debe realizar antes de generar el código fuente. Uno de los diagramas más utilizados para representar un software es el diagrama de clases. Este diagrama permite detallar todos los objetos que serán necesarios para que el producto final cumpla con las expectativas del cliente. Para tomar los requisitos del software existe otro tipo de herramienta que es muy útil para realizar esta tarea, la herramienta se denomina casos de uso. Estos sirven para anotar todo lo que el cliente mencione sobre el funcionamiento del software a realizar. El desarrollador deberá prestar atención a todo lo que el cliente necesite para el sistema. Los casos de uso son los más convenientes para redactar las necesidades o requisitos del sistema. A partir de todo lo mencionado por el cliente que utilizará el sistema, el analista empieza a reconocer los objetos que se deben generar para el posterior desarrollo del software. Por eso, el objetivo de este trabajo es crear una librería JavaSript que permita interpretar las descripciones de los casos de uso escritas con un lenguaje de símbolos que usa la herramienta TDDT4IoTS para marcar palabras que representan elementos del modelamiento orientado a objetos y generar la estructura de un diagrama de clases en formato JSON y XML, para que luego poder ser manipulado mediante otras librerías web especificas para dibujar otros tipos de diagramas, pero con la estructura que genera esta librería pueda conseguir el diagrama de clases de una forma automática. 

\textbf{Palabras claves:} Herramientas case, diagrama de clases, especificación de casos de uso, lenguaje para especificación casos de uso, ingeniería de software.
