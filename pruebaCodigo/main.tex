\documentclass[12pt,a4paper]{article}
\usepackage[utf8]{inputenc}
\usepackage[T1]{fontenc}
\usepackage{amsmath}
\usepackage{amssymb}
\usepackage{graphicx}
\usepackage[left=3.00cm, right=2.50cm, top=2.50cm, bottom=2.50cm]{geometry}
\usepackage[spanish]{babel}

\usepackage{listings}
\usepackage{listingsutf8}
\usepackage{color}

\definecolor{codegreen}{rgb}{0,0.6,0}
\definecolor{codegray}{rgb}{0.5,0.5,0.5}
\definecolor{backcolour}{rgb}{0.97, 0.97, 0.97}
\definecolor{codepurple}{rgb}{0.6, 0.4, 0.8}
%\definecolor{stringcolor}{rgb}{0.0, 0.5, 0.0}
\definecolor{stringcolor}{rgb}{0.97, 0.51, 0.47}

\lstdefinestyle{mystyle}{
	language=Python,
	keywordstyle=\color{codepurple}\bfseries,
	basicstyle={\footnotesize\ttfamily\color{black}},
	backgroundcolor=\color{backcolour},   
	commentstyle=\color{codegray},
	numberstyle=\ttfamily\color{black},
	stringstyle=\color{stringcolor},
	breakatwhitespace=false,         
	breaklines=true,                 
	captionpos=b,                    
	keepspaces=true,                 
	numbers=left,                    
	numbersep=10pt,                  
	showspaces=false,                
	showstringspaces=false,
	showtabs=false,                  
	tabsize=2, 
	frame=leftline
}

\lstset{literate=
	{á}{{\'a}}1
	{é}{{\'e}}1
	{í}{{\'i}}1
	{ó}{{\'o}}1
	{ú}{{\'u}}1
	{Á}{{\'A}}1
	{É}{{\'E}}1
	{Í}{{\'I}}1
	{Ó}{{\'O}}1
	{Ú}{{\'U}}1
	{ñ}{{\~n}}1
	{ü}{{\"u}}1
	{Ü}{{\"U}}1
	{\&}{{\&}}1
	{¡}{{!`}}1
	{¿}{{?`}}1
	{·}{{$\cdot$}}1
	{»}{{>>}}1
}

\lstset{style=mystyle}


\begin{document}
	\begin{lstlisting}[language=Python]
	{
	# Impartación de librerias para el funcionamiento
	import RPi.GPIO as GPIO
	import time
	import speech_recognition as sr
	import serial
	import struct
	from gtts import gTTS
	import os
	import unicodedata
	
	#Formato de texto
	trans_tab = dict.fromkeys(map(ord, u'\u0301\u0308'), None)
	#puerto_serie = serial.Serial('/dev/ttyAMA0',9600)
	#Inicializa puerto serial, velociad, paridad, puerto de tranmision raspberry
	puerto_serie = serial.Serial(port='/dev/ttyAMA0',baudrate=9600,parity=serial.PARITY_NONE,
	stopbits=serial.STOPBITS_ONE,
	bytesize=serial.EIGHTBITS,
	timeout=1) 
	# estructura para enviar hexadecimal
	k=struct.pack('B', 0xff)
	
	# funciones para imagen de cada letra, según posición de la A - Z
	def letraA(x,y):
	x=str(x)
	y=str(y)
	command = "pic "+ x +","+ y +",64"
	#command = 'pic 200,200,2'
	puerto_serie.write(command.encode())
	puerto_serie.write(k)
	puerto_serie.write(k)
	puerto_serie.write(k)
	def letraB(x,y):
	x=str(x)
	y=str(y)
	command = "pic "+ x +","+ y +",65"
	#command = 'pic 200,200,2'
	puerto_serie.write(command.encode())
	puerto_serie.write(k)
	puerto_serie.write(k)
	puerto_serie.write(k)	
	}
\end{lstlisting}
	
\end{document}